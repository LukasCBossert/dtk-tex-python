\documentclass[ngerman]{scrartcl}

\usepackage{libertine}
\usepackage{babel}
\usepackage{csquotes}
\usepackage{siunitx}
\usepackage{etoolbox}
\usepackage{keyval}
\usepackage{hyperref}
\usepackage{caption}
\usepackage[sort,nameinlink]{cleveref}%schönere Verweise

%----------------------
\newcommand\catalogueentry[1]{%
\begingroup
\setkeys{catalogue}{#1}%
\ifdef{\KVhouse}{\subsection{\KVhouse
			\ifdef{\KVlabel}{\label{\KVlabel}}{}}
			}{}%
\begin{labeling}{Beschreibung}
\ifdef{\KVdescription}{\item[Beschreibung] \KVdescription}{}%
\ifdef{\KVlocation}{\item[Verortung] \KVlocation}{}%
\ifdef{\KVinterior}{%
	\item[Ausstattung] \KVinterior %
	\ifboolexpr{bool{@KVinteriorM} or bool {@KVinteriorW} or bool {@KVinteriorS}}{%
	\begin{labeling}{Wandgemälde}
			\ifdef{\KVinteriorM}{\item[Mosaike:] \KVinteriorM}{}
			\ifdef{\KVinteriorW}{\item[Wandgemälde:] \KVinteriorW}{}
			\ifdef{\KVinteriorS}{\item[Statuen:] \KVinteriorS}{}
			\end{labeling}
		}{}}%
	{%
	\ifboolexpr{bool{@KVinteriorM} or bool {@KVinteriorW} or bool {@KVinteriorS}}{%
		\item[Ausstattung]%
	\begin{labeling}{Wandgemälde}	
				\ifdef{\KVinteriorM}{\item[Mosaike:] \KVinteriorM}{}
				\ifdef{\KVinteriorW}{\item[Wandgemälde:] \KVinteriorW}{}
				\ifdef{\KVinteriorS}{\item[Statuen:] \KVinteriorS}{}
			\end{labeling}
	}{}}%
  \ifdef{\KVsize}{\item[Größe] \SI{\KVsize}{\meter\squared}}{}%
  \ifdef{\KVfigures}{\item[Abbildungen] \labelcref{\KVfigures}}{}%
  \ifKVmark \item[Erwähnung] \expandafter\csname\KVlabel-pages\endcsname \fi
\end{labeling}
\endgroup
}

\makeatletter
\newbool{@KVinteriorM}%Mosaik
\newbool{@KVinteriorW}%Wandgemälde
\newbool{@KVinteriorS}%Statue
\define@key{catalogue}{house}{\gdef\KVhouse{#1}}
\define@key{catalogue}{label}{\def\KVlabel{#1}}
\define@key{catalogue}{description}{\def\KVdescription{#1}}
\define@key{catalogue}{location}{\def\KVlocation{#1}}
\define@key{catalogue}{size}{\def\KVsize{#1}}
\define@key{catalogue}{interior}{\def\KVinterior{#1}}%Innenausstattung
\define@key{catalogue}{interiorM}{\def\KVinteriorM{#1}\booltrue{@KVinteriorM}}
\define@key{catalogue}{interiorW}{\def\KVinteriorW{#1}\booltrue{@KVinteriorW}}
\define@key{catalogue}{interiorS}{\def\KVinteriorS{#1}\booltrue{@KVinteriorS}}
\define@key{catalogue}{figures}{\def\KVfigures{#1}}

\define@boolkey{catalogue}[KV]{mark}[true]{}

\newwrite\@catalog
\def\catalog@ref#1#2{%
    \expandafter\ifx\csname#1\endcsname\relax  % schon definiert??
    \@namedef{#1}{#2}%
    \else
    \expandafter\edef\csname#1\endcsname{\csname#1\endcsname, #2}%
    \fi}
\InputIfFileExists{\jobname.ctg}{}{}%
\immediate\openout\@catalog\jobname.ctg
\let\LabelCref\labelcref
\renewcommand\labelcref[1]{\expandafter\label@cref#1,,\@nil}
\def\label@cref#1,#2,#3\@nil{%
    \immediate\write\@catalog{\string\catalog@ref{#1-pages}{\arabic{page}}}
    \LabelCref{#1}%
    \ifx\relax#2\relax % kein label mehr da
    \def\next{}%
    \else 
    \def\next{\label@cref#2,#3\@nil}%
    \fi
    \next}
\AtEndDocument{\closeout\@catalog}
\renewcommand\labelcref[1]{\@cref{labelcref}{#1}}
\makeatother


\title{Meine wissenschaftliche Arbeit}
\date{}
\author{XY}

\crefmultiformat{subsection}{#2#1#3}{; #2#1#3}{; #2#1#3}{; #2#1#3}


\begin{document}
\maketitle

\section{Text}
In meiner Arbeit beschäftige ich mit drei Häusern in Pompeji 
(\labelcref{haus:M-Fabius-Rufus,haus:Haus-der-Venus,haus:Haus-des-Wilden-Ebers}).
Die Grundrisspläne: \cref{fig:M-Fabius-Rufus,fig:Haus-der-Venus,fig:Haus-des-Wilden-Ebers}.

\begin{figure}
\includegraphics[width=.45\textwidth]{example-image-a} 
\caption{Das Haus des Wilden Ebers.}
\label{fig:Haus-des-Wilden-Ebers}
\end{figure}

\clearpage
Dabei zeigt sich, dass das Haus des M. Fabius Rufus (\labelcref{haus:M-Fabius-Rufus}; 
\cref{fig:M-Fabius-Rufus}) das größte ist, 
und das Haus des Wilden Ebers (\labelcref{haus:Haus-des-Wilden-Ebers}; 
\cref{fig:Haus-des-Wilden-Ebers}) das kleinste.


\begin{figure}
\includegraphics[width=.5\textwidth]{example-image-b} 
\caption{Das Haus des M. Fabius Rufus.}
\label{fig:M-Fabius-Rufus}
\end{figure}

\clearpage
Alle Häuser liegen in der Regio VII, die Häuser ›Haus des Wilden Ebers‹ und ›Haus der Venus‹ (\cref{fig:Haus-der-Venus})zudem in der Insula 4.

(\labelcref{haus:M-Fabius-Rufus,haus:Haus-der-Venus,haus:Haus-des-Wilden-Ebers}).
\begin{figure}
\includegraphics[width=.5\textwidth]{example-image-c} 
\caption{Das Haus der Venus.}
\label{fig:Haus-der-Venus}
\end{figure}
\clearpage

\labelcref{haus:Haus-der-Venus}

\clearpage
\section{Katalogteil}
\renewcommand\thesection{Kat. \Alph{section}}
\catalogueentry{%
	house={Haus des M. Fabius Rufus},
	label={haus:M-Fabius-Rufus},
	size={172},
	description={Haus besteht aus mehreren Einzelgebäuden.},
	location={Regio VII, Insula 16, Eingang 17--22.},
	interior={Reicher Fundkomplex.},
	interiorM={S/W-Mosaik},
	interiorW={Dionysius mit einer Mänade, Narzissus und ein Cupido, Hercules und Deinira, etc.},
	interiorS={Bronzene Statue eines Epheben},
	figures={fig:M-Fabius-Rufus},
	mark,
}

\catalogueentry{%
	house={Haus des Wilden Ebers},
	label={haus:Haus-des-Wilden-Ebers},
	size={54},
	description={Renovierung nach Erdbeben 62\,n.\,Chr.},
	location={Regio VII, Insula 4, Eingang 48, 43},
	interiorM={S/W-Mosaik},
	interiorW={Venus, Leda und der Schwan, Ariadne und Theseus},
	figures = {fig:Haus-des-Wilden-Ebers},
	mark,
}

\catalogueentry{%
	house={Haus der Venus},
	label={haus:Haus-der-Venus},
	size={123},
	description={Schönes großes Haus},
	location={Regio VII, Insula 4, Eingang 12, 23},
	interior={Keine Innenausstattung},
	figures={fig:Haus-der-Venus},
	mark,
}


\end{document}
