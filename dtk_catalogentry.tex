\documentclass[ngerman]{dtk}
\ifluatex\else
\usepackage[utf8]{inputenc}
%\usepackage[latin9]{inputenc}
\fi
\usepackage{siunitx}
%\usepackage{etoolbox}
%\usepackage{keyval}
\usepackage{cleveref}




%----------------------
\newcommand{\catalogueentry}[1]{%
\begingroup\RaggedRight 
\setkeys{entrydetails}{#1}%
\ifdef{\KVhouse}{\section{\KVhouse
			\ifdef{\KVlabel}{\label{\KVlabel}}{}}
			}{}%
\begin{labeling}{Beschreibung}
%	\footnotesize %
%	\setlength{\itemsep}{0pt}%
%    \setlength{\parskip}{0pt}%
%    \setlength{\parsep}{0pt}%
\ifdef{\KVdescription}{\item[Beschreibung] \KVdescription}{}%
\ifdef{\KVlocation}{\item[Verortung] \KVlocation}{}%
\ifdef{\KVinterior}{%
	\item[Ausstattung] \KVinterior %
	\ifboolexpr{bool{@KVinteriorM} or bool {@KVinteriorW} or bool {@KVinteriorS}}{%
	\begin{labeling}{Wandgemälde}
%		\setlength{\itemsep}{0pt}%
%    		\setlength{\parskip}{0pt}%
%    		\setlength{\parsep}{0pt}%
			\ifdef{\KVinteriorM}{\item[Mosaike] \KVinteriorM}{}
			\ifdef{\KVinteriorW}{\item[Wandgemälde] \KVinteriorW}{}
			\ifdef{\KVinteriorS}{\item[Statuen] \KVinteriorS}{}
			\end{labeling}
		}{}}%
	{%
	\ifboolexpr{bool{@KVinteriorM} or bool {@KVinteriorW} or bool {@KVinteriorS}}{%
		\item[Ausstattung]%
	\begin{labeling}{Wandgemälde}	
%			\setlength{\itemsep}{0pt}
%	 		\setlength{\parskip}{0pt}%
%    			\setlength{\parsep}{0pt}%
				\ifdef{\KVinteriorM}{\item[Mosaike] \KVinteriorM}{}
				\ifdef{\KVinteriorW}{\item[Wandgemälde] \KVinteriorW}{}
				\ifdef{\KVinteriorS}{\item[Statuen] \KVinteriorS}{}
			\end{labeling}
	}{}}%
\ifdef{\KVsize}{	\item[Größe] \SI{\KVsize}{\meter\squared}}{}%
\ifdef{\KVliterature}{\item[Erwähnungen] \KVliterature}{}%		
\end{labeling}
\endgroup
}

\makeatletter
\newbool{@KVinteriorM}%Mosaik
\newbool{@KVinteriorW}%Wandgemälde
\newbool{@KVinteriorS}%Statue
\define@key{entrydetails}{house}{\def\KVhouse{#1}}
\define@key{entrydetails}{label}{\def\KVlabel{#1}}
\define@key{entrydetails}{description}{\def\KVdescription{#1}}
\define@key{entrydetails}{location}{\def\KVlocation{#1}}
\define@key{entrydetails}{size}{\def\KVsize{#1}}
\define@key{entrydetails}{interior}{\def\KVinterior{#1}}%Innenausstattung
\define@key{entrydetails}{interiorM}{\def\KVinteriorM{#1}\booltrue{@KVinteriorM}}
\define@key{entrydetails}{interiorW}{\def\KVinteriorW{#1}\booltrue{@KVinteriorW}}
\define@key{entrydetails}{interiorS}{\def\KVinteriorS{#1}\booltrue{@KVinteriorS}}
\define@key{entrydetails}{literature}{\def\KVliterature{#1}}
\makeatother


\begin{document}
\title{Integration von Python in \LaTeX\ am Beispiel von Katalogeinträgen}
\Author{Uwe}{Ziegenhagen}{}
\Author{Lukas C.}{Bossert}%
			{Cranachstr.~24\\
			12157 Berlin\\
			\Email{lukas@digitales-altertum.de}}
\maketitle



Viele Dissertationen in der Archäologie enthalten am Ende einen Katalog, 
in dem die untersuchten Daten in einem bestimmten System aufgeschlüsselt präsentiert werden.
Dies können Bilder, Bohrproben oder Architekturen sein.

Eine händische Erstellubg über section oder subsection und bspw mehreren items ist weder effizent und nur bei wenigen Einträgen einsetzbar.
Es muss zudem berücksichtigt werden, dass nicht auf alle katalogeinträge alle vorgegebne Kategorien passwn, sodass ggf. Kategorien leer bleiben würden und dann im Katalogeintrag nicht auftauchen sollen.
Darüberhinaus sollen alle Einträge immer gleich formatiert sein und ihr aussehen global verändert werden können.

Ein Katalogeintrag soll alle Seitenzahlen enthalten, auf denen der Katalogeintrag im Haupttext erwähnt wurde.
Für diese Anforderung erinnerte ich (Lukas) mich an einen Beitrag, den Uwe Ziegenhagen in dtk 1/2015 vorgestellt hat. 
Darin ging es jedoch leider nur darum lediglich eine Erwähnung aus dem Haupttext im Katalogeintrag anzugeben.

Meine Anfrage bei Uwe hat ergeben, dass die in dtk 1/2015 beschriebene vorgehendweise nur eine seitenzahl ausgeben kann.
Allerdings schlug er vor mittels Python das Problem zu lösen und alle Seitenzahlen zusammenfassend im jeweiligen katalogeintrag ausgeben zu können.

Um gemeinsam an dem hybriden Konstrukt von LaTeX und Python wurde die Plattform github ausgewählt, die unabhängig der Programmiersprache einen exzellenten Austausch und eine detaillierte Versuonskontrolle bereithält.


\subsection{Katalogeintrag}



Eine Lösung bietet ...\footnote{Basierend auf der Idee vorgestellt auf: \url{http://tex.stackexchange.com/a/254336/98739}}

Kat. \ref{haus:M-Fabius-Rufus}





\begin{lstlisting}[style=number]
\catalogueentry{%
	house={Haus des M. Fabius Rufus},
	label={haus:M-Fabius-Rufus},
	size={172},
	description={Haus besteht aus mehreren Einzelgebäuden.},
	location={Regio VII, Insula 16, Eingang 17--22.},
	interior={Reicher Fundkomplex.},
	interiorM={S/W-Mosaik},
	interiorW={Dionysius mit einer Mänade, Narzissus und ein Cupido, Hercules und Deinira, etc.},
	interiorS={Bronzene Statue eines Epheben},
}
\end{lstlisting}


\catalogueentry{%
	house={Haus des M. Fabius Rufus},
	label={haus:M-Fabius-Rufus},
	size={172},
	description={Haus besteht aus mehreren Einzelgebäuden.},
	location={Regio VII, Insula 16, Eingang 17--22.},
	interior={Reicher Fundkomplex.},
	interiorM={S/W-Mosaik},
	interiorW={Dionysius mit einer Mänade, Narzissus und ein Cupido, Hercules und Deinira, etc.},
	interiorS={Bronzene Statue eines Epheben},
}
\newpage
Weiteres Beispiel mit teilweiser Innenausstattung:

\begin{lstlisting}[style=number]
\catalogueentry{%
	house={Haus des Wilden Ebers},
	label={haus:Wilden-Ebers},
	size={54},
	description={Renovierung nach Erdbeben 62\,n.\,Chr.},
	location={Regio VII, Insula 4, Eingang 48, 43},
	interiorM={S/W-Mosaik},
	interiorW={Venus, Leda und der Schwan, Ariadne und Theseus},
}
\end{lstlisting}

\catalogueentry{%
	house={Haus des Wilden Ebers},
	label={haus:Wilden-Ebers},
	size={54},
	description={Renovierung nach Erdbeben 62\,n.\,Chr.},
	location={Regio VII, Insula 4, Eingang 48, 43},
	interiorM={S/W-Mosaik},
	interiorW={Venus, Leda und der Schwan, Ariadne und Theseus},
}


\newpage
Notwendige Pakete
\begin{lstlisting}[style=noNumber]
\usepackage{etoolbox}
\usepackage{keyval}
\usepackage{cleveref}
\usepackage{siunitx}
\end{lstlisting}

\begin{lstlisting}[style=number]
\newcommand{\catalogueentry}[1]{%
\begingroup\RaggedRight 
\setkeys{entrydetails}{#1}%
\ifdef{\KVhouse}{\section{\KVhouse
			\ifdef{\KVlabel}{\label{\KVlabel}}{}}
			}{}%
\begin{labeling}{Beschreibung}
\ifdef{\KVdescription}{\item[Beschreibung] \KVdescription}{}%
\ifdef{\KVlocation}{\item[Verortung] \KVlocation}{}%
\ifdef{\KVinterior}{%
	\item[Ausstattung] \KVinterior %
	\ifboolexpr{bool{@KVinteriorM} or bool {@KVinteriorW} or bool {@KVinteriorS}}{%
	\begin{labeling}{Wandgemälde}
			\ifdef{\KVinteriorM}{\item[Mosaike] \KVinteriorM}{}
			\ifdef{\KVinteriorW}{\item[Wandgemälde] \KVinteriorW}{}
			\ifdef{\KVinteriorS}{\item[Statuen] \KVinteriorS}{}
			\end{labeling}
		}{}}%
	{%
	\ifboolexpr{bool{@KVinteriorM} or bool {@KVinteriorW} or bool {@KVinteriorS}}{%
		\item[Ausstattung]%
	\begin{labeling}{Wandgemälde}	
				\ifdef{\KVinteriorM}{\item[Mosaike] \KVinteriorM}{}
				\ifdef{\KVinteriorW}{\item[Wandgemälde] \KVinteriorW}{}
				\ifdef{\KVinteriorS}{\item[Statuen] \KVinteriorS}{}
			\end{labeling}
	}{}}%
\ifdef{\KVsize}{	\item[Größe] \SI{\KVsize}{\meter\squared}}{}%
\ifdef{\KVliterature}{\item[Erwähnungen] \KVliterature}{}%		
\end{labeling}
\endgroup
}

\makeatletter
\newbool{@KVinteriorM}%Mosaik
\newbool{@KVinteriorW}%Wandgemälde
\newbool{@KVinteriorS}%Statue
\define@key{entrydetails}{house}{\def\KVhouse{#1}}
\define@key{entrydetails}{label}{\def\KVlabel{#1}}
\define@key{entrydetails}{description}{\def\KVdescription{#1}}
\define@key{entrydetails}{location}{\def\KVlocation{#1}}
\define@key{entrydetails}{size}{\def\KVsize{#1}}
\define@key{entrydetails}{interior}{\def\KVinterior{#1}}%Innenausstattung
\define@key{entrydetails}{interiorM}{\def\KVinteriorM{#1}\booltrue{@KVinteriorM}}
\define@key{entrydetails}{interiorW}{\def\KVinteriorW{#1}\booltrue{@KVinteriorW}}
\define@key{entrydetails}{interiorS}{\def\KVinteriorS{#1}\booltrue{@KVinteriorS}}
\define@key{entrydetails}{literature}{\def\KVliterature{#1}}
\makeatother
\end{lstlisting}



\subsection{python}

\subsection{Integration}
\end{document}